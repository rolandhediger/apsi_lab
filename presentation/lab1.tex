\documentclass{beamer}
\usetheme{Warsaw}
\usepackage[utf8x]{inputenc}
\usepackage[german]{babel}
\usepackage{color}
\usepackage{xcolor}
\usepackage{listings}
\usepackage{caption}
\DeclareCaptionFont{white}{\color{white}}
\DeclareCaptionFormat{listing}{\colorbox{gray}{\parbox{\textwidth}{#1#2#3}}}
\captionsetup[lstlisting]{format=listing,labelfont=white,textfont=white}
\lstset{
 language=Java,
 basicstyle=\footnotesize\ttfamily, % Standardschrift
 numbers=left,               % Ort der Zeilennummern
 numberstyle=\tiny,          % Stil der Zeilennummern
 stepnumber=5,              % Abstand zwischen den Zeilennummern
 numbersep=5pt,              % Abstand der Nummern zum Text
 tabsize=2,                  % Groesse von Tabs
 extendedchars=true,         %
 breaklines=true,            % Zeilen werden Umgebrochen
 frame=b,         
 %commentstyle=\itshape\color{LightLime}, Was isch das? O_o
 %keywordstyle=\bfseries\color{DarkPurple}, und das O_o
 basicstyle=\small,
 stringstyle=\color[RGB]{42,0,255}\ttfamily, % Farbe der String
 keywordstyle=\color[RGB]{127,0,85}\ttfamily, % Farbe der Keywords
 commentstyle=\color[RGB]{63,127,95}\ttfamily, % Farbe des Kommentars
 showspaces=false,           % Leerzeichen anzeigen ?
 showtabs=false,             % Tabs anzeigen ?
 xleftmargin=17pt,
 framexleftmargin=17pt,
 framexrightmargin=5pt,
 framexbottommargin=4pt,
 showstringspaces=false      % Leerzeichen in Strings anzeigen ?        
}
\newcommand\Fontvi{\fontsize{6}{7.2}\selectfont}
\begin{document}
\title{FHNW APSI Lab 1 : Wichtigkeit von $2^n$}   
\author{Roland Hediger, Jonas Schwammberger} 

\date{\today} 

\frame{\titlepage} 

\begin{frame}
\section*{Übersicht}
\tableofcontents
\end{frame} 

\begin{frame}
\section{Generation der Briefe}
\large \textbf{Datenstruktur:}
\begin{itemize}
 \item Statische Teile des Briefes bleiben dasselbe, jeder dynamische Teil hat ein Platzhalter, \texttt{\{ \# 1\}}
 \item 32 Plathalter, 2 Möglichkeiten : Insgesamt $2^{32}$ mögliche Briefe \textbf{(mit gleicher Konto Nr)}
 \item Datenstruktur : \texttt{HashMap<Integer,ArrayList<String>}
 \item  Alle möglichkeiten für Platzhalter sind hardcodiert.
\end{itemize}
\end{frame}

\begin{frame}[fragile]
\begin{lstlisting}
 private int createVariation(int combination, String file) {

		String tmpfile = new String(file);
		for (int i = 0; i < 32; i++) {
			int bit = combination & 1;
			String placeHolderString = "{#" + Integer.toString(i) + "}";
			ArrayList<String> combinationsForPlaceHolder = map.get(i);
			tmpfile = tmpfile.replace(placeHolderString,
					combinationsForPlaceHolder.get(bit));
			combination >>>= 1;
		}

		return hash.createHash(tmpfile.getBytes());
	}
\end{lstlisting}
\end{frame}
\begin{frame}
 \begin{itemize}
  \item \texttt{Integer.MAX = $2^{32}$}
  \item Jeder Brief Variante als Integer zwischen 0 und Integer.MAX
  \item Möglichkeit für Platzhalter gegeben durch 1 oder 0 in  \textbf{Binärdarstellung des Integers}
  \item Nächste Möglichkeit für Plathalter des Briefes wird mittels \textbf{vorzeichenloses Bitshift} erreicht
 \end{itemize}

\end{frame}


\begin{frame}
\section{Hashfunktion}
\begin{enumerate}
 \item Preprocessing : Nehme Byte Array und füge Padding hinzu damit der Länge durch 8 teilbar ist.
 \item Hasherzeugung :
 \item IV definiert als 8 Byte Array mit 0s
\end{enumerate}
\textbf{Für jeder Block}
\begin{enumerate}
 \item BufferedBlockCipher initializieren mit IV
 \item Output Array für DES (16 Byte) initializieren mithilfe von BufferedBlockCipher.
 \item Bytes verarbeiten für diese Iteration.(Processbytes
 \item XOR ``linke und rechte'' Blöcke (i, und i+8 im DES Output Array) und dann XOR mit IV.
 \item Swap verfahren : Erzeugte hash wird zu neue IV. (Alles in \texttt{byte[]} Format bis jetzt)
 \item Postprocessing : Hash um 32 bits schieben, und XOR mit Integer.reverse von hash. (\texttt{int} Format).
\end{enumerate}
\end{frame}

\begin{frame}[allowframebreaks,fragile]
\section{Integration von Bouncycastle}
\begin{lstlisting}
 	private long create(byte[] input) {
		BlockCipher engine = new DESEngine();
		@SuppressWarnings("deprecation")
		BufferedBlockCipher cipher = new PaddedBlockCipher(engine);
		byte[] desOut = new byte[16];
		byte[] hash = new byte[8];
		byte[] previousHash = iv.clone();

		for (int i = 0; i < input.length; i += 8) {
			KeyParameter p = new KeyParameter(previousHash);
			cipher.init(true, p);
			desOut = new byte[cipher.getOutputSize(8)];
			int outputLen = cipher.processBytes(input, i, 8, desOut, 0);

			try {
				cipher.doFinal(desOut, 0);

				// xor magix
				for (int j = 0; j < hash.length; j++)
					hash[j] = (byte) ((desOut[j] ^ desOut[j + 8]) ^ previousHash[j]);

				// swap
				byte[] tmp = hash;
				hash = previousHash;
				previousHash = tmp;

			} catch (CryptoException ce) {
				System.err.println(ce);
			}

		}

		ByteBuffer buffer = ByteBuffer.wrap(previousHash);
		buffer.order(ByteOrder.LITTLE_ENDIAN);
		return buffer.getLong();
	}

\end{lstlisting}

\end{frame}

\begin{frame}
\section{Kollisionssuche}
\begin{itemize}
 \item 2 Mengen gebildet : 1 mit alle Variationen basierend auf originalem KontoNr, und eine genau so aber mit gefälschte KontoNr.
\end{itemize}
\begin{block}{}
 Statt variationen für alle Kombinationen zu machen, wählen wir 2 beliebige Kombinationen.
\end{block}

\begin{itemize}
 \item Hashen von den Briefvarianten , prüfen auf gleicheit.
 \item Ausgabe der Variantennummern und den dazugehörigen Hashes.
\end{itemize}
\begin{block}{}
 Falls keine Gleichheit gefunden wird, wählen wir weitere beliebige Kombinationen in Blocken von 1024 aus der ganzen Menge.
\end{block}

\end{frame}
\begin{frame}
 \section*{Ende}
 \huge Danke für Ihre Aufmerksamkeit.
\end{frame}

\end{document}

