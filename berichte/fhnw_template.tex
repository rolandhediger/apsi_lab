\documentclass[12pt]{scrartcl}
 \usepackage{fancyhdr, graphicx}
 \usepackage[german]{babel}
 \usepackage[scaled=0.92]{helvet}
 \usepackage{enumitem}
 \usepackage{parskip}
 \usepackage{lastpage} % for getting last page number
 \renewcommand{\familydefault}{\sfdefault}
 
 \fancypagestyle{firststyle}{ %Style of the first page
 \fancyhf{}
 \fancyheadoffset[L]{0.6cm}
 \lhead{
 \includegraphics[scale=0.8]{./fhnw_ht_e_10mm.jpg}}
 \renewcommand{\headrulewidth}{0pt}
 \lfoot{APSI Lab 1}
 \rfoot{Roland Hediger, Jonas Schwammberger}
}

\fancypagestyle{documentstyle}{ %Style of the rest of the document
 \fancyhf{}
 \fancyheadoffset[L]{0.6cm}
\lhead{
 \includegraphics[scale=0.8]{./fhnw_ht_e_10mm.jpg}}
 \renewcommand{\headrulewidth}{0pt}
 \lfoot{\thepage\ / \pageref{LastPage} }
}

\pagestyle{firststyle} %different look of first page
 
\title{ %Titel
APSI Lab 1
\vspace{0.2cm}
}

 \begin{document}
 \maketitle
 \thispagestyle{firststyle}
 \pagestyle{firststyle}
 \begin{abstract}
 \begin{center}
  put abstract here  
 \end{center}
 \vspace{0.5cm}
\hrulefill
\end{abstract}

 \pagestyle{documentstyle}
 \tableofcontents
 \pagebreak
\section{Intro}
Ihre Aufgabe besteht darin, sogenannte Kollisionen im Hash-Verfahren zu suchen, d.h. Änderungen
im Originaltext, die den gleichen Hashwert liefern: $ h(m_{orig}) = h(m_{fake})$. Wie Sie vielleicht
bereits bemerkt haben, handelt es sich um eine praktische Anwendung des bekannten
Geburtstagsparadoxons, das Sie in der Mathematik bzw. in der Kryptologie kennengelernt
haben.

\section{Hashfunktion}

\section{Variationserzeugung}
Für diese Aufgabe haben wir $2^{32}$ verschiedene Kombinationsmöglichkeiten pro Mail. Diese Kombinationen haben wir in einem Integer codiert. Jedes Bit repräsentiert einen Index eines Platzhalters. Zum Beispiel: Das zweite Bit steht auf 0, dann wird das Wort "vom Herzen" eingesetzt.

\subsection{Datenstrukturen}
Alle Platzhaltertexte haben wir in einer Hashmap mit \"Platzhalter\" index als Schlüssel und einem Arraylist der grösse 2 für die Texte.

\section{Kollisionsdetektion}
\subsection{Strategie}
Wir können zwischen zwei Strateien unterscheiden. Entweder, wir generieren alle Original und Fake Variationen linear (beginnend bei 0), oder wir benutzen einen Random-Generator für die Original und die Fake Strings.

\subsection{Datenstrukturen}

 \end{document}