\documentclass[10pt]{scrartcl}

%Math
\usepackage{amsmath}
\usepackage{amsfonts}
\usepackage{amssymb}
\usepackage{amsthm}
\usepackage{ulem}
\usepackage{stmaryrd} %f\UTF{00FC}r Blitz!

%PageStyle
\usepackage[ngerman]{babel} % deutsche Silbentrennung
\usepackage[utf8]{inputenc} 
\usepackage{fancyhdr, graphicx}
\usepackage[scaled=0.92]{helvet}
\usepackage{enumitem}
\usepackage{parskip}
\usepackage[a4paper,top=2cm]{geometry}
\setlength{\textwidth}{17cm}
\setlength{\oddsidemargin}{-0.5cm}
\usepackage[scaled=0.92]{helvet}
\usepackage{lastpage} % for getting last page number
\renewcommand{\familydefault}{\sfdefault}


% Shortcommands
\newcommand{\Bold}[1]{\textbf{#1}} %Boldface
\newcommand{\Kursiv}[1]{\textit{#1}} %Italic
\newcommand{\T}[1]{\text{#1}} %Textmode
\newcommand{\Nicht}[1]{\T{\sout{$ #1 $}}} %Streicht Shit durch

%Arrows
\newcommand{\lra}{\leftrightarrow} 
\newcommand{\ra}{\rightarrow}
\newcommand{\la}{\leftarrow}
\newcommand{\lral}{\longleftrightarrow}
\newcommand{\ral}{\longrightarrow}
\newcommand{\lal}{\longleftarrow}
\newcommand{\Lra}{\Leftrightarrow}
\newcommand{\Ra}{\Rightarrow}
\newcommand{\La}{\Leftarrow}
\newcommand{\Lral}{\Longleftrightarrow}
\newcommand{\Ral}{\Longrightarrow}
\newcommand{\Lal}{\Longleftarrow}

% Code listenings
\usepackage{color}
\usepackage{xcolor}
\usepackage{listings}
\usepackage{caption}
\DeclareCaptionFont{white}{\color{white}}
\DeclareCaptionFormat{listing}{\colorbox{gray}{\parbox{\textwidth}{#1#2#3}}}
\captionsetup[lstlisting]{format=listing,labelfont=white,textfont=white}
\lstdefinestyle{JavaStyle}{
 language=Java,
 basicstyle=\footnotesize\ttfamily, % Standardschrift
 numbers=left,               % Ort der Zeilennummern
 numberstyle=\tiny,          % Stil der Zeilennummern
 stepnumber=5,              % Abstand zwischen den Zeilennummern
 numbersep=5pt,              % Abstand der Nummern zum Text
 tabsize=2,                  % Groesse von Tabs
 extendedchars=true,         %
 breaklines=true,            % Zeilen werden Umgebrochen
 frame=b,         
 %commentstyle=\itshape\color{LightLime}, Was isch das? O_o
 %keywordstyle=\bfseries\color{DarkPurple}, und das O_o
 basicstyle=\footnotesize\ttfamily,
 stringstyle=\color[RGB]{42,0,255}\ttfamily, % Farbe der String
 keywordstyle=\color[RGB]{127,0,85}\ttfamily, % Farbe der Keywords
 commentstyle=\color[RGB]{63,127,95}\ttfamily, % Farbe des Kommentars
 showspaces=false,           % Leerzeichen anzeigen ?
 showtabs=false,             % Tabs anzeigen ?
 xleftmargin=17pt,
 framexleftmargin=17pt,
 framexrightmargin=5pt,
 framexbottommargin=4pt,
 showstringspaces=false      % Leerzeichen in Strings anzeigen ?        
}

 
\fancypagestyle{firststyle}{ %Style of the first page
\fancyhf{}
\fancyheadoffset[L]{0.6cm}
\lhead{
\includegraphics[scale=0.8]{./fhnw_ht_e_10mm.jpg}}
\renewcommand{\headrulewidth}{0pt}
\lfoot{Institute of computer science,\linebreak www.fhnw.ch }
}

\fancypagestyle{documentstyle}{ %Style of the rest of the document
\fancyhf{}
\fancyheadoffset[L]{0.6cm}
\lhead{
\includegraphics[scale=0.8]{./fhnw_ht_e_10mm.jpg}}
\renewcommand{\headrulewidth}{0pt}
\lfoot{\thepage\ / \pageref{LastPage} }
}

\pagestyle{firststyle} %different look of first page
 
\title{ %Titel
Titel
\vspace{0.2cm}
\Large more titel }

 \begin{document}
 \maketitle
 \thispagestyle{firststyle}
 \pagestyle{firststyle}
 \begin{abstract}
 \begin{center}
  put abstract here  
 \end{center}
 \vspace{0.5cm}
\hrulefill
\end{abstract}

 \pagestyle{documentstyle}
 \tableofcontents
 \pagebreak
\section{Robust Programming}
Folgende
\subsection{Model}
\subsubsection{XXS, SQL-Injection und Validierung}

\subsubsection{Schutz der Model-Klasse vor schädlichem Missbrauch}
Die schwierigkeit einer korrekten Login-implementation liegt darin, den Ablauf so zu gestalten dass nur dieser Ablauf zu einer validen Änderung führt. Heisst nur beim vorgesehenen Ablauf können nur korrekte Daten geschrieben werden, alles andere darf die Datenbank nicht verändern. So sind beispielsweise setters auf das Passwort problematisch. Zusammen mit der $.save()$ Methode könnte ein Fehlerhafter oder bösartiger Code das Passwort eines beliebigen Users überschreiben. Usernamen und ID sind ebenfalls Felder, auf die nur lesend zugegriffen werden soll.

Umsetzung:


\subsubsection{Postleitzahl Validierung}
Die Postleitzahl wird wie verlangt, durch einen externen Service überprüft. Wir überprüfen nur, ob die Postleitzahl existiert und nicht ob die angegebene Stadt mit der Postleitzahl übereinstimmt. 

\subsection{Controller}
Der Controller wurde so simpel wie möglich gehalten, 

\subsubsection{SSL}

\section{Weitere Mögliche Massnahmen}

\subsection{DOS-Attacke}

 \end{document}