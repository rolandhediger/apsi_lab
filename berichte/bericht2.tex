\documentclass[10pt]{scrartcl}

%Math
\usepackage{amsmath}
\usepackage{amsfonts}
\usepackage{amssymb}
\usepackage{amsthm}
\usepackage{ulem}
\usepackage{stmaryrd} %f\UTF{00FC}r Blitz!

%PageStyle
\usepackage[ngerman]{babel} % deutsche Silbentrennung
\usepackage[utf8]{inputenc} 
\usepackage{fancyhdr, graphicx}
\usepackage[scaled=0.92]{helvet}
\usepackage{enumitem}
\usepackage{parskip}
\usepackage[a4paper,top=2cm]{geometry}
\setlength{\textwidth}{17cm}
\setlength{\oddsidemargin}{-0.5cm}
\usepackage[scaled=0.92]{helvet}
\usepackage{lastpage} % for getting last page number
\renewcommand{\familydefault}{\sfdefault}


% Shortcommands
\newcommand{\Bold}[1]{\textbf{#1}} %Boldface
\newcommand{\Kursiv}[1]{\textit{#1}} %Italic
\newcommand{\T}[1]{\text{#1}} %Textmode
\newcommand{\Nicht}[1]{\T{\sout{$ #1 $}}} %Streicht Shit durch

%Arrows
\newcommand{\lra}{\leftrightarrow} 
\newcommand{\ra}{\rightarrow}
\newcommand{\la}{\leftarrow}
\newcommand{\lral}{\longleftrightarrow}
\newcommand{\ral}{\longrightarrow}
\newcommand{\lal}{\longleftarrow}
\newcommand{\Lra}{\Leftrightarrow}
\newcommand{\Ra}{\Rightarrow}
\newcommand{\La}{\Leftarrow}
\newcommand{\Lral}{\Longleftrightarrow}
\newcommand{\Ral}{\Longrightarrow}
\newcommand{\Lal}{\Longleftarrow}

% Code listenings
\usepackage{color}
\usepackage{xcolor}
\usepackage{listings}
\usepackage{caption}
\DeclareCaptionFont{white}{\color{white}}
\DeclareCaptionFormat{listing}{\colorbox{gray}{\parbox{\textwidth}{#1#2#3}}}
\captionsetup[lstlisting]{format=listing,labelfont=white,textfont=white}
\lstdefinestyle{JavaStyle}{
 language=Java,
 basicstyle=\footnotesize\ttfamily, % Standardschrift
 numbers=left,               % Ort der Zeilennummern
 numberstyle=\tiny,          % Stil der Zeilennummern
 stepnumber=5,              % Abstand zwischen den Zeilennummern
 numbersep=5pt,              % Abstand der Nummern zum Text
 tabsize=2,                  % Groesse von Tabs
 extendedchars=true,         %
 breaklines=true,            % Zeilen werden Umgebrochen
 frame=b,         
 %commentstyle=\itshape\color{LightLime}, Was isch das? O_o
 %keywordstyle=\bfseries\color{DarkPurple}, und das O_o
 basicstyle=\footnotesize\ttfamily,
 stringstyle=\color[RGB]{42,0,255}\ttfamily, % Farbe der String
 keywordstyle=\color[RGB]{127,0,85}\ttfamily, % Farbe der Keywords
 commentstyle=\color[RGB]{63,127,95}\ttfamily, % Farbe des Kommentars
 showspaces=false,           % Leerzeichen anzeigen ?
 showtabs=false,             % Tabs anzeigen ?
 xleftmargin=17pt,
 framexleftmargin=17pt,
 framexrightmargin=5pt,
 framexbottommargin=4pt,
 showstringspaces=false      % Leerzeichen in Strings anzeigen ?        
}

 
\fancypagestyle{firststyle}{ %Style of the first page
\fancyhf{}
\fancyheadoffset[L]{0.6cm}
\lhead{
\includegraphics[scale=0.8]{./fhnw_ht_e_10mm.jpg}}
\renewcommand{\headrulewidth}{0pt}
\lfoot{Institute of computer science,\linebreak www.fhnw.ch }
}

\fancypagestyle{documentstyle}{ %Style of the rest of the document
\fancyhf{}
\fancyheadoffset[L]{0.6cm}
\lhead{
\includegraphics[scale=0.8]{./fhnw_ht_e_10mm.jpg}}
\renewcommand{\headrulewidth}{0pt}
\lfoot{\thepage\ / \pageref{LastPage} }
}

\pagestyle{firststyle} %different look of first page
 
\title{ %Titel
Apsi
\vspace{0.2cm}
\Large Lab 2 }

 \begin{document}
 \maketitle
 \thispagestyle{firststyle}
 \pagestyle{firststyle}
 \begin{abstract}
 \begin{center}

 \end{center}
 \vspace{0.5cm}
\hrulefill
\end{abstract}

 \pagestyle{documentstyle}
 \tableofcontents
 \pagebreak
\section{Datenbankabsicherung}

 
 
\section{Robust Programming}

\subsection{Model}
\subsubsection{XXS, SQL-Injection und Validierung}
Validerung wurde mittels Regex vorgenommen. Wir benutzen sogennante ``Clean Strings''. m.a.w Regex Ausdrücke die ungültige Zeichen nicht erlauben. Die Validierung erzwingen wir indem die $save()$ Methode nur in die Datenbank schreibt, wenn vorher die Funktion $validate()$ aufgerufen wurde.

Gegen SQL Injection wird geschützt mithilfe von ``Prepared Statements''. Diese SQL Befehle erlauben nur die Parameter 
vom Benutzer im Query an bestimmten stellen. Die Parameter werden automatisch von SQL-Injections \"gesäubert\".
 
Schutz gegen XSS wird durch von unsere Cleanstrings gewährleisted, diese verhindern Eingaben wie $<script>$ Tags und andere javascript symbole.

\subsubsection{Schutz der Model-Klasse vor schädlichem Missbrauch}
Die schwierigkeit einer korrekten Login-implementation liegt darin, den Ablauf so zu gestalten dass nur dieser Ablauf zu einer validen Änderung führt. Heisst nur beim vorgesehenen Ablauf können nur korrekte Daten geschrieben werden, alles andere darf die Datenbank nicht verändern. So sind beispielsweise setters auf das Passwort problematisch. Zusammen mit der $.save()$ Methode könnte ein Fehlerhafter oder bösartiger Code das Passwort eines beliebigen Users überschreiben. Usernamen und ID sind ebenfalls Felder, auf die nur lesend zugegriffen werden soll.

Umsetzung:
Um das Model korrekt umzusetzen wurden zwei Konstruktoren implementiert. Ein Konstruktor nimmt einen Benutzernamen und Passwort entgegen. Dieser ist zuständig, einen neue erstellten Benutzer zu repräsentieren. Ein zweiter Konstruktor nimmt eine ID und ein Passwort entgegen. ID und Benutzername werden nur 

\subsubsection{Postleitzahl Validierung}
Die Postleitzahl wird wie verlangt, durch einen externen Service überprüft. Die Website fragt post.ch ab,ob eine gewisse Postleitzahl existiert. Wenn diese nicht existiert, enthält die HTML Antwort den String \"Keine PLZ Gefunden\" und wir lassen diese PLZ nicht zu. Wir überprüfen nur, ob die Postleitzahl existiert und nicht ob die angegebene Stadt mit der Postleitzahl übereinstimmt. 

\subsection{Controller}
Der Controller wurde so simpel wie möglich gehalten. Dieser ist nur zuständig, die richtigen Methoden des Models aufzurufen, die Request/Response parameter zu lesen und befüllen und den User auf die richtigen Seiten weiterzuleiten.. Es wurde darauf geachtet, dass das Model nur erlaubte Änderungen in die Datenbank schreibt.


\section{Sicherung mit SSL}
Im folgenden Abschitt wird der Code gelistet, welcher wir benutzt haben um ein Zertifikat zu erzeugen und dann wie man Tomcat mit dem Zertifikat einzurichtet.

\subsection{SSL CA Erzeugung}
\begin{verbatim}
  OPENSSL=ca.cnf openssl req -x509 -nodes -days 3650     -newkey rsa:2048 -out apsica/certs/ca.pem     -outform PEM 
-keyout ./apsica/private/ca.key

\end{verbatim}

\subsection{Zertifikat erzeugung}
\begin{verbatim}
 Certificate:
    Data:
        Version: 3 (0x2)
        Serial Number: 1 (0x1)
    Signature Algorithm: sha1WithRSAEncryption
        Issuer: C=CH, ST=AG, L=Brugg, O=FHNW, N=apsi.fhnw.ch/emailAddress=apsi@rolandh.tk
        Validity
            Not Before: Dec 18 08:10:19 2013 GMT
            Not After : Dec 18 08:10:19 2014 GMT
        Subject: CN=apsi.fhnw.ch, ST=AG, C=CH/emailAddress=apsi@rolandh.tk, O=FHNW
        Subject Public Key Info:
            Public Key Algorithm: rsaEncryption
                Public-Key: (2048 bit)
                Modulus:
                    00:a4:43:a2:c2:93:3f:a8:52:de:f1:7b:b1:00:16:
                    da:c9:84:ac:b9:1e:60:08:b9:66:db:62:0b:5a:8a:
                    9f:17:d8:b9:49:c8:3a:2c:31:7a:ff:11:0e:aa:88:
                    c8:77:cf:b9:da:6e:bb:b7:94:57:82:64:c6:2f:ca:
                    26:b3:d8:bf:4e:0b:11:0a:cf:3a:81:a4:71:3d:47:
                    ff:b0:22:0f:85:4f:28:05:f4:52:0d:bb:f4:62:1f:
                    08:3c:3e:35:fe:10:e3:1f:13:9b:5c:07:90:a3:32:
                    9d:fb:00:7d:ed:7a:f0:69:ca:56:d0:b0:21:32:2b:
                    66:90:c3:c2:c9:0a:a2:0f:ac:34:7d:20:93:2d:fb:
                    73:02:78:d4:1d:b2:7e:6d:6a:89:5a:fb:09:04:94:
                    b2:41:7f:29:1b:09:8c:14:6a:f8:ec:c0:f7:a1:38:
                    6b:a4:ec:0c:fe:9d:28:6a:e6:64:a2:cc:16:11:89:
                    32:2e:c8:e2:52:2c:1c:6d:73:e7:32:9c:ee:32:c0:
                    0d:3e:4f:1b:3c:95:2f:20:2f:6f:cf:89:7f:82:74:
                    1d:36:0c:46:64:41:8d:98:9c:fd:15:ff:2b:83:ad:
                    8e:7d:0a:f3:8e:42:ac:ec:9d:a6:a4:40:16:02:84:
                    ce:38:da:e1:d2:f8:e7:e7:d1:b7:5a:bb:cb:90:99:
                    c3:1b
                Exponent: 65537 (0x10001)
        X509v3 extensions:
            X509v3 Basic Constraints: 
                CA:FALSE
    Signature Algorithm: sha1WithRSAEncryption
         4c:c0:a7:51:25:fa:0a:61:4c:60:30:1d:3b:d8:0d:00:c7:44:
         73:81:7f:2b:aa:27:65:7f:4c:82:08:6b:26:2f:a9:37:30:38:
         23:58:89:17:16:48:42:45:1c:de:a1:04:11:e5:65:f0:dd:af:
         e6:2a:e7:e4:cb:b1:89:e7:ef:83:c9:8c:7e:fc:42:0e:27:76:
         6a:bc:db:68:af:6d:a5:78:d2:2f:e7:75:49:22:3a:91:5e:26:
         69:87:ee:58:5b:9f:53:c5:5d:9d:1c:55:c7:1e:ea:af:fa:b0:
         e3:6d:8c:63:d6:36:07:b4:30:4b:e0:80:83:8c:fd:cc:e7:de:
         5a:3f:ef:16:35:6f:32:11:bc:0c:d2:f4:0a:d6:ee:79:05:0a:
         d6:e3:36:e0:f8:68:4a:3a:ed:25:be:5f:e0:56:24:d5:1f:5e:
         68:0e:9b:3c:d1:88:d3:f0:a1:54:a2:ce:5f:c6:c0:a2:79:28:
         00:8e:47:cd:1e:6d:54:35:05:37:0f:cf:b2:c9:0e:96:b2:84:
         69:36:1c:b0:99:39:3f:dc:1b:d8:8e:35:15:22:80:dc:71:55:
         bb:34:19:da:a0:5c:84:6a:c2:e6:e5:00:8a:06:63:56:d1:93:
         d6:4a:45:c5:79:3f:76:6c:59:e7:dc:9b:fd:39:69:c8:f2:f2:
         86:76:aa:69
\end{verbatim}

Dieses Zertifikat ist mit einem CSR verbunden und nachdem wir es mit unserer CA unterzeichnet haben, ist es im Keystore integriert.

\subsection{Tomcat Server Config}
\begin{verbatim}
 <Connector port="8443" maxThreads="150" scheme="https" secure="true" SSLEnabled="true" 
keystoreFile="ssl/apsi.jks" keystorePass="apsiKennwort" clientAuth="false" keyAlias="apsi" 
sslProtocol="TLS"/>
\end{verbatim}

Web.xml: \\
\begin{verbatim}
 <security-constraint>

<web-resource-collection>

<web-resource-name>RattleBits</web-resource-name>

<url-pattern>/*</url-pattern>

</web-resource-collection>

<user-data-constraint>

<transport-guarantee>CONFIDENTIAL</transport-guarantee>

</user-data-constraint>

</security-constraint>
\end{verbatim}


\section{Weitere Mögliche Massnahmen}

\subsection{Session Timeouts}
Sessions sollten einen geeigneten Timeoutwert haben. Dieser muss jedoch auf das Benutzerverhalten abgestimmt werden.

\subsection{Brute Force Logins}
Gescheiterte Loginversuche müssen in der Datenbank abgespeichert werden. Wenn die Anzahl der Fehlversuche einen Grenzwert überschreitet, muss dieser Benutzer eine Zeit abwarten bevor er sich wieder einloggen kann. 
 \end{document}